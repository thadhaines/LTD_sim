\documentclass[12pt]{article}
\usepackage[english]{babel}
\usepackage[utf8]{inputenc}

\usepackage{geometry}
\geometry{
	letterpaper, 
	portrait, 
	top=.75in,
	left=.8in,
	right=.75in,
	bottom=.5in		} 	% Page Margins
	
%% additional packages for nice things
\usepackage{amsmath} 	% for most math
\usepackage{commath} 	% for abs
\usepackage{lastpage}	% for page count
\usepackage{amssymb} 	% for therefore
\usepackage{graphicx} 	% for image handling
\usepackage{wrapfig} 	% wrap figures
\usepackage[none]{hyphenat} % for no hyphenations
\usepackage{booktabs} 	% enhanced table qualities
\usepackage{array} 		% for >{} column characterisctis
\usepackage{physics} 	% for easier derivative \dv....
\usepackage{tikz} 		% for graphic@!
\usepackage{circuitikz} % for circuits!
\usetikzlibrary{arrows.meta} % for loads
\usepackage[thicklines]{cancel}	% for cancels
\usepackage{xcolor}		% for color cancels
\usepackage[per-mode=fraction]{siunitx} % for si units and num
\usepackage{fancyhdr} 	% for header
\usepackage{comment}	% for ability to comment out large sections
\usepackage{multicol}	% for multiple columns using multicols
\usepackage[framed,numbered]{matlab-prettifier} % matlab sytle listing
\usepackage{marvosym} 	% for boltsymbol lightning
\usepackage{pdflscape} 	% for various landscape pages in portrait docs.
\usepackage{float}
\usepackage{fancyvrb}	% for Verbatim (a tab respecting verbatim)
\usepackage{enumitem}	% for [resume] functionality of enumerate

%% package config 
\sisetup{output-exponent-marker=\ensuremath{\mathrm{E}}} % for engineer E
\renewcommand{\CancelColor}{\color{red}}	% for color cancels
\lstset{aboveskip=2pt,belowskip=2pt} % for more compact table
\def\arraystretch{1.4} % adjust size of arrays
%\arraycolsep=1.4pt\def
\setlength{\parindent}{0cm} % Remove indentation from paragraphs
\setlength{\columnsep}{0.5cm}
\lstset{
	style      = Matlab-editor,
	basicstyle = \ttfamily\footnotesize, % if you want to use Courier - not really used?
}
\renewcommand*{\pd}[3][]{\ensuremath{\dfrac{\partial^{#1} #2}{\partial #3}}} % for larger pd fracs
\renewcommand{\real}[1]{\mathbb{R}\left\{ #1 \right\}}	% for REAL symbol
\newcommand{\imag}[1]{\mathbb{I}\left\{ #1 \right\}}	% for IMAG symbol
\definecolor{m}{rgb}{1,0,1}	% for MATLAB matching magenta
	
%% custom macros
\newcommand\numberthis{\addtocounter{equation}{1}\tag{\theequation}} % for simple \numberthis command
\newcommand{\equal}{=} % so circuitikz can have an = in the labels
\newcolumntype{L}[1]{>{\raggedright\let\newline\\\arraybackslash\hspace{0pt}}m{#1}}
\newcolumntype{C}[1]{>{\centering\let\newline\\\arraybackslash\hspace{0pt}}m{#1}}
\newcolumntype{R}[1]{>{\raggedleft\let\newline\\\arraybackslash\hspace{0pt}}m{#1}}

%% Header
\pagestyle{fancy} % for header stuffs
\fancyhf{}
\rhead{Thad Haines \\ Page \thepage\ of \pageref{LastPage}}
\chead{Long-Term Dynamic Simulation Thoughts \\ February 11, 2019}
\lhead{Research \\ }
% spacing
\headheight 29 pt
\headsep 6 pt

\begin{document}

\begin{comment}
page 1 -> separte weekly talking points file.

\paragraph{Random Thoughts:}
\begin{enumerate}
\item Numerical integration at beginning of simulation may require extra care as to avoid out of index errors. -Maybe not because python can do limited negative indexing.
\end{enumerate}
\end{comment}

\paragraph{Introduction:}
Power systems are large and complicated structures composed of many connected objects. Each power system object has a functional purpose and reacts to various changes differently. These objects are often modeled mathematically as transfer functions or state equations that alter certain parameters (states) of the objects themselves, or the system as a whole, over time. \\

A wide variety of software exists to model transient (fast) reactions of power systems. These softwares involve small time steps so that models can have fast time constants and capture the oscillatory nature of power system responses. While these models have their place, it may be desirable to calculate slower, more long-term dynamic responses to gradual perturbances, or disturbances, such as wind ramps, daily load cycles, or balancing authority type situations. \\

General Electric's Positive Sequence Load Flow (PSLF) is a commercial power systems analysis software package that is regarded as an industry standard. Many working and accepted system models exist that are compatible with PSLF. Additionally, GE has created a .NET interface to their software for a more customized experimentation and simulation process. %Luckily, there seems to be enough functionality to perform LTD. \\

\paragraph{Goals:}
The first goal of this project is to develop software that performs long-term dynamic (LTD) simulations using available PSLF systems, dynamic data, and modified/custom dynamic models. An Agent Based modeling approach will be taken to accommodate a wide variety of power systems and enable the independent and modular development of features. \\

%This approach enables tong term events will be programmed perturbances Additional This software solution will be used to simulate system responses to programmable long-term events. 
Once complete, the specialized functionality of the software framework will attempt to facilitate investigating system reactions that may have been previously overlooked, computationally heavy, or programmatically difficult to simulate via other methods.
%Since power systems vary greatly, a flexible simulation environment must be able to handle many unique system scenarios. 
\paragraph{System Under Investigation:}
When performing such LTD simulations on multi-machine power systems, certain simplifying assumptions will be made:
\begin{enumerate}
	\item The system is synchronized.
	\item The system has only 1 frequency and it is altered by the aggregate PU swing equation
	\[ \dot{f}_{sys} = \dfrac{1}{2H_{sys} } \left( \dfrac{P_{acc, sys} }{f_{sys}(t)} - D_{sys}\Delta f{sys}(t)  \right)  \]
	\item System damping $D_{sys}$ is ignored (set to 0).
	\item Time steps of 1 second (or larger) will allow for ignoring transient rotor windings and generator damping time constants. 
	\item Existing dynamic models with time constants smaller than the system time step (1 second) must be reworked to avoid mathematical errors.
	\item Accelerating power will be distributed to system generators participating in the inertial response according to a ratio of object inertias.
\end{enumerate}

\pagebreak
\paragraph{Conceptual Model:}
Ironpython (IPY) will take file locations of PSLF .sav (power system information file) and .dyd (PSLF dynamic model information) files that will be used to generate a python 'mirror' of a PSLF power system. 
IPY will communicate to PSLF via the .NET API to gather and/or update system data (states) and run PSLF power-flow calculations. 
Between power-flow calculations, custom dynamic models will react to the most recent power-flow data to change associated IPY 'mirror' and PSLF states.  
Finally IPY will save/export collected data and exports a .mat file by running a Python 3 script.\\

Long-term simulations my be performed by the GE dynamic simulation (PSDS) to validate LTD simulation results. It is believed that the custom LTD code will perform the desired simulations much faster than PSDS and will exhibit reasonable accuracy to solve engineering problems. 
\paragraph{Conceptual Model Details of Note:}
\begin{enumerate}
\item Software requirements: PSLF with middleware.dll, Ironpython, Python 3, and Visual Studio with Python Development Tools (plus GitHub extension).

\item The PSLF .sav and .dyd files contain data that describes any given system.

\item Ironpython code will establish 'mirror' of the PSLF system with data available from solved PSLF case data, system parameters from the API, and dynamic model information parsed from the .dyd file.

\item Ironpython code will interact with PSLF to solve PSLF power flow numerous times and record data as system states change.

\item A routine of Model and Agent steps will be developed that is capable of:
	\begin{enumerate}
		\item Calculating system frequency response from any remaining system Pacc
		\item Calculating current accelerating power  (Pm-Pe-Perturbance) where:
		\begin{enumerate}
			\item Pm will be collected from generators in the system
			\item Pe will be a summation of the previous time steps solved power flow
			\item Perturbance data will be summed per time step from all perturbance agents.
		\end{enumerate}
		\item Distributing electric power among generators in python 'mirror' according to machine inertia:
		% Will require some more time logic/explanation
		\[ P_{e,i}(t) = P_{e,i}(t-1)-\Delta P_{acc,sys}(t)\dfrac{H_{i}}{H_{sys}} \]
		Note that this will only affect generators participating in the inertial response.
		\item Updating PSLF system with newly calculated generator powers and voltages.
		\item Performing PSLF power flow solution of system.
		\item Checking slack bus error and repeating power distribution as required while also ensuring convergence.
		
		\item Calculating object dynamic responses in a prioritized manner.
		%\item Possibly performing one more power flow so system is balanced after mechanical response?
		\item Logging states and increasing system time.
		%\item Once electric power is distributed satisfactorily, mechanical power must be calculated by the dynamic response of each generator.
		%\item This dynamic response will calculate new mechanical power and voltage levels of the system that will then be used to calculate change in system frequency via the aggregate swing equation.
		%\item Changes to state will then be logged, system time increased, and the process repeated ad infinitum.
	\end{enumerate}
\item After a simulation, desired data may be saved as a pickled python object, dictionary, or MATLAB .mat file.
\item Quick display functionality may be built into the python mirror to view system activity immediately following execution using the \verb|matplotlib.pyplot| library.
\end{enumerate}
 
\pagebreak
\paragraph{Simulation Model:}
Using an agent based approach, each object in the simulation will be activated in various substeps per each time step. These steps will be carried out in a loop once the system has been initialized for a predetermined amount of time / time steps unless an error occurs or the system fails to converge.\vspace{1em}\\ 
It is assumed that transfer function models may be simplified to state space arrays, or something equivalent, that can be populated by specific model parameters. Once solved, i.e. slope is found, other numerical techniques can be employed to predict the next system state. Alternatively, Ironpython, or available .NET, ode solvers may be used if applicable and available.
\subparagraph{Substeps :}(Performed each time step/ main model step)
\begin{enumerate}
	\item Perterubance step: A perturbance object will be programmed / configured before the simulation and attached to the environment. These agents will alter the power into/out of other system agents (such as generators and loads). Once this step is complete a total system Perturbance can be calculated.
	\item[2 a.] Distribution Step: Once $P_{acc}$ is calculated, it will be distributed to the system. These new values will be updated in the PSLF system.
	% , or python mirror, depending on the most current object. (Probably update python, then save to PSLF, though the order may change depending on which process seems smarter / easier)
	\item[2 b.] Power Flow Solution : Using the PSLF powerflow solver allows the system to find balance. If slack error is too great, the error will be sent back to the beginning of the Distribution Step to be processed  by non-Slack generators again. Note: Since this computation happens at the end of the Distribution Step, it may not be a unique step and more of a "while: slack error $>$ some error tolerance".\\
	Once the slack bus is below error, Electric power has been adequately distributed and will be sent to the Dynamics step as well as being summed for the next time step.
	\item[3.] Dynamics step: Using current and history values, agent related models are simulated to calculate responses to the change in accelerating power and frequency. This will probably involve state space solutions to ODEs (transfer functions) and numerical integration (PSLF uses Adams-Bashforth, a different method may have to be used due to timestep size). Additionally, prioritization may have to occur so that proper 'model chains' may be executed in a realistic manner.
\end{enumerate}


\pagebreak

\begin{minipage}[t]{.5\textwidth}
\subparagraph{Agents:} Each agent keeps track of its own history data and references to PSLF.  PSLF references are numerous and may only apply to certain agents. Some examples are: id, longId, Area, Zone, Bus number, Bus name, Scan bus number, External Number
\\
\textbf{Core Agents}
\begin{itemize}
	\item Bus: Keeps track of Voltage Magnitude and Angle
	\begin{enumerate}
		\item Useful for data collection 
		\item As each bus steps, it steps each attached Generator (or other LTD element) 
		\item bus voltage and angle will change after each individual element step.
	\end{enumerate}
	\item Line / Branch: to keep track of line current 
	\item Loads: For perturbance access and history of P, Q, and status.
	\item Generators: Slack, non-slack 
	\begin{enumerate}
		\item Track Pe, Pm, Q, and status.
		\item Slack gen keeps track of expected Pe for error correction applications
		\item Will have a participation flag \\(default = 1)
		\item will have dynamic model list (probably prioritized), that will execute each step
	\end{enumerate} 
	\item Perturbance: to simulate known/possible events
	\begin{enumerate}
		\item Various Types: steps, ramps, sines
		\item Applied to generators, loads, or other controllable element (breakers?).
		\item Control of P, Q, V, status
		\item Additional type: noise (adaptive or set range) - to add an element of randomness to simulation. Possibly un-necessary.
	\end{enumerate} 
	
\end{itemize}

\end{minipage} \hspace{1 em}\begin{minipage}[t]{.5\textwidth}
%\columnbreak
\textbf{Additional Agents}
\begin{itemize}
	\item Balancing Authority
	\item Power Plant Supervisor
	\item Microgrid / Battery
	\item Area
	\item Shunt \& SVD
\end{itemize}
\subparagraph{Model:} Initializes PSLF, mirror, and agents. Keeps track of global system parameters and dictates step and substep operations/sequence.
\begin{itemize}
	\item Required Inputs:
	\begin{enumerate}
		\item File locations for PSLF
		\item time step
		\item simulation end time
		\item Slack error tolerance
	\end{enumerate}
	\item Optional Inputs:
	\begin{enumerate}
		\item H of system (else calculated)
		\item D of system (else calculated)
		\item Debug flag
	\end{enumerate}
	\item Keeps current value and hisotry of:
	\begin{enumerate}
		\item time
		\item system frequency
		\item power summations:  $P_{acc} \ P_{e} \ P_{m} \ P_{perturb}$
	\end{enumerate}
\end{itemize}
\paragraph{Additional Required Functions / Content:}
\begin{itemize}
	\item .dyd parser / PSLF dynamic model builder
	\item Dynamic go between (for model init)
	\item LTD model library
	\item Data Collector / Exporter (MATLAB?)
	\item Data Presenter (grapher)
\end{itemize}
\end{minipage}

\begin{comment}
\begin{landscape}
	\begin{center}
		\includegraphics[height=\textheight]{stajcar_flowchar_edit.png}
	\end{center}
\end{landscape}
\begin{landscape}
	\begin{center}
		\includegraphics[height=\textheight]{LTD_basis.pdf}
	\end{center}
\end{landscape}
\end{comment}

\end{document}
